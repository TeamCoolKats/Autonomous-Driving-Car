\documentclass[12pt]{article}
\usepackage[utf8]{inputenc}
\usepackage{enumitem}
\usepackage{geometry}
\geometry{margin=1in}
\title{Software Requirements Specification\\\vspace{0.3em}\large Autonomous Driving Without Lane Marks}
\author{Humayra Rashid, Daniel Gomez, Henry Xiong, Rodrigo Valdez}
\date{\today}

\begin{document}
\maketitle

\section*{Revision History}
\begin{tabular}{|l|l|l|}
\hline
\textbf{Name} & \textbf{Date} & \textbf{Reason for Changes} \\
\hline
Humayra Rashid, Daniel Gomez,\\ Henry Xiong, Rodrigo Valdez & \today & Initial Draft \\
\hline
\end{tabular}

\newpage

\section{Introduction}
\subsection{Purpose}
This document outlines the software requirements for an autonomous driving system capable of navigating without clear lane markings. It includes all functional and non-functional requirements needed for development.

\subsection{Intended Audience and Reading Suggestions}
This document is intended for software engineers, project managers, and testers. Section 4 is most relevant to developers and testers, while sections 1 and 2 provide context for stakeholders.

\subsection{Product Scope}
The project involves designing and implementing a software system that enables autonomous vehicles to drive on roads with unclear or absent lane markings. The system will process real-time sensor input to determine navigation paths.

\subsection{Definitions, Acronyms, and Abbreviations}
\begin{itemize}[noitemsep]
  \item ADA - Autonomous Driving Algorithm
  \item PM - Processing Module
  \item RM - Response Module
  \item SIM - Simulator
\end{itemize}

\subsection{References}
No academic papers were cited. Project built on top of publicly available open-source driving simulators.

\section{Overall Description}
\subsection{System Analysis}
The system addresses the challenge of autonomous driving without lane marks. Major hurdles include accurate environmental perception and safe decision-making. Solutions involve a modular software pipeline with pre-processing and response logic.

\subsection{Product Perspective}
This system is a standalone software solution that can be integrated into toy-scale or full-scale vehicles equipped with sensors. It does not depend on any existing proprietary systems.

\subsection{Product Functions}
\begin{itemize}[noitemsep]
  \item Interpret visual data to identify drivable paths.
  \item Process sensor input to detect obstacles.
  \item Compute steering and speed commands.
\end{itemize}

\subsection{User Classes and Characteristics}
\begin{itemize}[noitemsep]
  \item Developers - configure modules and debug performance.
  \item Testers - evaluate system behavior in simulated and real-world conditions.
\end{itemize}

\subsection{Operating Environment}
The software will run on Raspberry Pi or Android systems in the prototype and on a desktop computer for simulation.

\subsection{Design and Implementation Constraints}
Must be lightweight enough to run on embedded systems. Should work without GPS or high-resolution maps.

\subsection{User Documentation}
A developer's guide and simulation test plan will be included.

\subsection{Assumptions and Dependencies}
Assumes availability of a camera, a motor controller, and basic Linux-based system.

\subsection{Apportioning of Requirements}
Advanced lane merging and multi-car coordination will be postponed for later versions.

\section{External Interface Requirements}
\subsection{User Interfaces}
CLI and logs only; no GUI. Possible voice input in later versions.

\subsection{Hardware Interfaces}
Interacts with sensors (camera, ultrasonic), motors, and microcontrollers.

\subsection{Software Interfaces}
Built on open-source libraries like OpenCV. Will communicate with a control daemon on the prototype.

\subsection{Communications Interfaces}
Uses Bluetooth for prototype communication and file-based interfaces in simulator.

\section{Requirements Specification}
\subsection{Functional Requirements}
\textbf{SM - Sensor Module}
\begin{itemize}[noitemsep]
  \item SM.1 The system shall collect visual data from the front-facing camera.
  \item SM.2 The system shall detect obstacles and send data to the controller.
\end{itemize}

\textbf{CM - Controller Module}
\begin{itemize}[noitemsep]
  \item CM.1 The system shall receive sensor input and pass it to the app module.
  \item CM.2 The system shall forward control commands to the motor.
\end{itemize}

\textbf{AM - Application Module}
\begin{itemize}[noitemsep]
  \item AM.1 The system shall manage the processing and response pipeline.
  \item AM.2 The system shall communicate results back to the controller.
\end{itemize}

\subsection{External Interface Requirements}
Described in Section 3.

\subsection{Logical Database Requirements}
No persistent data storage required. All data processed in real time.

\subsection{Design Constraints}
Must run in under 200MB RAM. Execution must remain under 100ms per loop.

\section{Other Nonfunctional Requirements}
\subsection{Performance Requirements}
95\% of commands must be generated in under 100ms.

\subsection{Safety Requirements}
Prototype must stop movement on sensor failure.

\subsection{Security Requirements}
Bluetooth pairing must be encrypted.

\subsection{Software Quality Attributes}
\begin{itemize}[noitemsep]
  \item Modularity
  \item Real-time responsiveness
  \item Portability to embedded devices
\end{itemize}

\subsection{Business Rules}
Only trusted developers can change the driving policy logic.

\section{Legal and Ethical Considerations}
Prototype testing will be done only in safe, closed environments. Code will be open-source for educational use only.

\end{document}
